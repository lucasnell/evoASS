
\section*{Methods}


\subsection*{Model overview}

We used a simple model of competition to evaluate how competition affects outcomes of
community assembly and trait evolution.
Each of $n$ species has $q$ traits that affect competition.
All traits and species are symmetrical.
Increases in traits benefit the species via reduced
effects of both interspecific and intraspecific competition;
they also incur a cost by decreasing the species' population growth rate.
In both the costs and benefits, trait effects are concave functions that,
combined, ensure that one fitness peak exists for each trait.
Traits can also have non-additive trade-offs that either increase or decrease the cost
associated with increasing multiple traits compared to increasing just one.
Trait evolution in one species also affects competition experienced by others.
This can be either conflicting or non-conflicting:
When conflicting, increasing traits in one species increase the effects of competition
for all other species.
When non-conflicting, all competitors' competition is reduced when traits increase in
one species.

We analyzed the model for 2--4 traits.
Below, we use matrix notation that applies to any number of traits.
Matrices (indicated by bold face) are multiplied using matrix---not
element-wise---multiplication,
and ${}^{\textrm{T}}$ represents matrix transposition.

We used a discrete-time, modified Lotka--Volterra competition model similar to
that by \citet{Northfield:2013if}.
In it, species $i$ has a $1 \times q$ matrix of traits ($\mathbf{V}_i$), and
its per-capita growth---equivalent to fitness---is

\begin{equation} \label{eq:fitness}
    F_{i} = \exp \left\{ r_i(\mathbf{V}_i) - \alpha_{ii}(\mathbf{V}_i) N_i - \sum_{j \ne i}^{n}{
        \alpha_{ij}(\mathbf{V}_i, \mathbf{V}_j) N_j}  \right\}\textrm{,}
\end{equation}

\noindent where $N_i$ is the population density of  species $i$.
The parameter $r(\mathbf{V}_i)$ describes how a trait increase decreases the growth rate:

\begin{equation} \label{eq:growth-rate}
\begin{split}
    r(\mathbf{V}_i) &= r_0 - f ~ \mathbf{V}_i ~ \mathbf{C} ~ \mathbf{V}_{i}^{\textrm{T}} \\
    \mathbf{C} &= \begin{pmatrix}
        1      & \eta_{12} & \ldots & \eta_{1q} \\
        \eta_{21} & 1      & \ldots & \eta_{2q} \\
        \vdots & \vdots & \ddots & \vdots \\
        \eta_{q1} & \eta_{q2} & \ldots & 1      \\
        \end{pmatrix}
    \textrm{,}
\end{split}
\end{equation}

\noindent where $r_0$ is the baseline growth rate,
$f$ is the cost of increasing traits on the growth rate, and
$\eta_{k,l}$ is the non-additive trade-off of increasing both the
$k$\textsuperscript{th} and $l$\textsuperscript{th} traits.
When $\eta > 0$, increasing multiple traits incurs an extra cost.
Non-additive trade-offs are symmetrical (i.e., $\eta_{k,l} = \eta_{l,k}$ for all
$l$ and $k$).
This and other relationships between traits and other parameters are of the form
$v^2 \propto X$ for parameter $X$, so $v = z$ is equivalent to $v = -z$.
Thus, traits are not allowed to be $< 0$.


%The term $\alpha(\mathbf{V}, \mathbf{N}, i)$ in
%equation \ref{eq:fitness} represents how traits affect
%intraspecific and interspecific competition.
%The overall effect of competition is given by
%
%\begin{equation} \label{eq:competition}
%\begin{split}
%    \alpha(\mathbf{V}, \mathbf{N}, i) &=
%        \alpha_0 ~ \textrm{e}^{- \mathbf{V}_i
%            \mathbf{V}_i^{\textrm{T}} }
%        \left[ N_i ~ +
%        \sum_{j \ne i}^{n}{
%            N_j ~
%            \textrm{e}^{- d ~ \mathbf{V}_j \mathbf{V}_j^{\textrm{T}} } }
%        \right]
%    \textrm{.}
%\end{split}
%\end{equation}
%
%where
%$\alpha_0$ is the base density dependence and
%$N_i$ is the abundance of species $i$.
%Parameter $d$ determines how the evolution of traits in one species affects
%competition experienced by others:
%When $d < 0$, increase of $\mathbf{V}_i$ increases the effects of competition in
%all other species.
%Alternatively, when $d > 0$, $\mathbf{V}_i$ increase decreases the effects of
%competition on all other species.
%Thus $d < 0$ causes conflicting and $d > 0$ causes non-conflicting
%evolution \citep{Northfield:2013if}.


The terms $\alpha_{ii}(\mathbf{V}_i)$ and $\alpha_{ij}(\mathbf{V}_i, \mathbf{V}_j)$
in equation \ref{eq:fitness} represent how trait increases decrease the effects
of intraspecific and interspecific competition, respectively.
Competition effects are given by

\begin{equation} \label{eq:competition}
\begin{split}
    \alpha_{ii}(\mathbf{V}_i) &= \alpha_0 ~\exp \left\{- \mathbf{V}_i
        \mathbf{V}_i^{\textrm{T}} \right\} \\
    \alpha_{ij}(\mathbf{V}_i, \mathbf{V}_j) &= \alpha_0 ~\exp \left\{
        - \mathbf{V}_i \mathbf{V}_i^{\textrm{T}} -
        \mathbf{V}_j \mathbf{D} \mathbf{V}_j^{\textrm{T}} \right\} \\
    \mathbf{D} &= \begin{pmatrix}
        d_1     & \ldots    & 0 \\
        \vdots  & \ddots    & \vdots \\
        0       & \ldots    & d_q
        \end{pmatrix}
	\textrm{,}
\end{split}
\end{equation}

\noindent where $\alpha_0$ is the base density dependence.
Matrix $\mathbf{D}$ contains parameters that determine how evolution of traits
in one species affects competition experienced by others:
When $d_k < 0$, increases in $v_{i,k}$ decrease the
effect of competition on species $i$, but increase it in all others.
Alternatively, when $d_k > 0$, increases in $v_{i,k}$ decrease the effect of
competition on all species including itself.
Thus $d < 0$ causes conflicting and $d > 0$ causes non-conflicting evolution
\citep{Northfield:2013if}.


If we combine the above equations and simplify them, we get

\begin{equation} \label{eq:fitness-full}
\begin{split}
    F_{i} &= \exp \left\{
        r_0 - f ~ \mathbf{V}_i ~ \mathbf{C} ~ \mathbf{V}_{i}^{\textrm{T}} -
        \alpha_0 ~\textrm{e}^{- \mathbf{V}_i \mathbf{V}_i^{\textrm{T}} } \mathbf{\Omega}_{i}
        \right\} \\
        \mathbf{\Omega}_i &\equiv N_i +
            \sum_{j \ne i}^{n}{ N_j \textrm{e}^{ - \mathbf{V}_j \mathbf{D} \mathbf{V}_j^{\textrm{T}} } }
        \textrm{,}
\end{split}
\end{equation}

\noindent where $\mathbf{\Omega}_i$ represents the community abundance scaled
for the effect each species' trait values has on competition
experienced by species $i$.
Hereafter we will refer to $\mathbf{\Omega}_i$ as the scaled community size.




% \subsection*{Adaptive dynamics}
%
% We started simulations with a single competitive species with trait values set to zero.
% We tracked species population densities through time using equation \ref{eq:fitness} and
% considered a species extinct if its density fell below $10^{-4}$.
% Species produced daughter species with a probability of 0.01 per species per time step.
% We generated daughter-species trait values from normal distributions with means of the
% mother trait values and standard deviations of $\sigma_{d}$.


\subsection*{Quantitative genetics}

We used quantitative genetics to explore the geometry of the fitness landscapes for
competing species.
We assume that all traits in $\mathbf{V}_i$ represent species means and that their
among-individual distributions are symmetrical with additive genetic variance
of $\sigma^2_i$.
Assuming also that $\sigma^2_i$ is relatively small
\citep{Iwasa:1991eo,Abrams:2001va,Abrams:1993cr}, traits at time $t+1$ are

\begin{equation} \label{eq:trait-change}
    \mathbf{V}_{i,t+1} = \mathbf{V}_{i,t} + \left( \frac{1}{F_i}
        \frac{\partial F_i}{\partial \mathbf{V}_{i}} \right) \sigma^2_i
    \textrm{.}
\end{equation}

% We started these simulations with 100 species.
% Each had starting trait values generated from a standard normal distribution.
% We tracked densities through time using equation \ref{eq:fitness} and
% considered a species extinct if its density fell below $10^{-4}$.
% Equation \ref{eq:trait-change} dictated how traits changed.

To determine the stability of ending points (trait values and abundances of surviving
competitor(s)), we computed the $nq \times nq$ Jacobian matrices of first derivatives
for the traits of each species:

\begin{equation} \label{eq:jacobian}
    \begin{pmatrix}
        \frac{\partial \mathbf{V}_{1,t+1}}{\partial \mathbf{V}_{1,t}} & \cdots &
            \frac{\partial \mathbf{V}_{n,t+1}}{\partial \mathbf{V}_{1,t}} \\
        \vdots & \ddots & \vdots \\
        \frac{\partial \mathbf{V}_{1,t+1}}{\partial \mathbf{V}_{n,t}} & \cdots &
            \frac{\partial \mathbf{V}_{n,t+1}}{\partial \mathbf{V}_{n,t}}
    \end{pmatrix}
    \textrm{.}
\end{equation}

\noindent We then computed the primary eigenvalue of this matrix ($\lambda$).
We considered a state stable when $\lambda < 1$,
neutrally stable when $\lambda = 1$,
and unstable when $\lambda > 1$.

Full analytical solutions are found in appendix A.


\subsection*{Code}

We simulated models using a combination of R \citep{RCoreTeam:2019wf} and
C++ via the Rcpp and RcppArmadillo packages
\citep{Eddelbuettel:2014ad,Eddelbuettel:2013if,Sanderson:2016cs}.
We double-checked our derivations by simulating 100 datasets
(4 species with 3 traits each) and computing derivatives using the Theano Python
library's automatic differentiation \citep{TheanoDevelopmentTeam:2016uc}.
All code can be found on GitHub
% (\texttt{https://github.com/lucasnell/sauron}).
(links are available from the journal office).

