\section*{Abstract}


Recent studies have shown the usefulness of incorporating trait evolution into models
of competition in the context of community assembly.
In these and other models, competition is often treated implicitly, as a side-effect of,
for example, interactions with external resources.
Here, we used a simple community model where members' evolving traits directly affect
competition experienced by all members of the community.
We evaluated how evolving competition affects outcomes of community assembly,
including the conditions under which coexistence occurs and the equilibria at
which communities ultimately arrive.
Treating competition as an abstraction also allowed us to examine complex
patterns relating to community equilibria.
We find that coexistence occurs only when evolution is non-conflicting or neutral
and that community "saturation" affects the shape of fitness landscapes.
Trait non-additive effects influence whether more than one evolutionarily
stable community exists, and complex patterns such as alternative stable
states and neutrally stable attractor manifolds are possible.

