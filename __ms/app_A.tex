\section*{Appendix A: Matrix derivatives in quantitative genetics equations}

\renewcommand{\thefigure}{A\arabic{figure}}
\renewcommand{\theequation}{A\arabic{equation}}
\renewcommand{\thetable}{A\arabic{table}}
\setcounter{equation}{0}
\setcounter{figure}{0}
\setcounter{table}{0}


As in the main text, $^{\textrm{T}}$ indicates transposition,
multiplication between matrices is always matrix multiplication, and
bold face indicates a matrix.
Also note that both $\mathbf{C}$ and $\mathbf{D}$ are symmetrical,
so $\mathbf{C} + \mathbf{C}^{\textrm{T}} = 2 \; \mathbf{C}$ and
$\mathbf{D} + \mathbf{D}^{\textrm{T}} = 2 \; \mathbf{D}$.


\subsection*{Trait change}

From the main text, we know that

\begin{equation*}
\begin{split}
    F_{i,t+1} &= \exp \left\{
        r_0 - f \; \mathbf{V}_{i,t} \; \mathbf{C} \; \mathbf{V}_{i,t}^{\textrm{T}} -
        \alpha_0 \;\textrm{e}^{- \mathbf{V}_{i,t} \mathbf{V}_{i,t}^{\textrm{T}} } \mathbf{\Omega}_{i,t}
        \right\} \\
    \mathbf{V}_{i,t+1} &= \mathbf{V}_{i,t} + \left( \frac{1}{F_{i,t+1}}
        \frac{\partial F_{i,t+1}}{\partial \mathbf{V}_{i,t}} \right) \sigma^2_i
    \textrm{.}
\end{split}
\end{equation*}


The partial derivative of fitness in relation to traits for species $i$ is


\begin{equation*}
\begin{split}
    \frac{\partial F_{i,t+1}}{\partial \mathbf{V}_{i,t}} &=
        \exp \left\{
            r_0
            - f \mathbf{V}_{i,t} \mathbf{C} \mathbf{V}_{i,t}^{\textrm{T}}
            - \alpha_0  \mathbf{\Omega}_{i,t} \,
                \textrm{e}^{- \mathbf{V}_{i,t} \mathbf{V}_{i,t}^{\textrm{T}}}
        \right\}
        \frac{\partial \!
            \left(
                r_0
                - f \; \mathbf{V}_{i,t} \; \mathbf{C} \; \mathbf{V}_{i,t}^{\textrm{T}}
                - \alpha_0 \; \mathbf{\Omega}_{i,t} \;
                    \textrm{e}^{- \mathbf{V}_{i,t} \mathbf{V}_{i,t}^{\textrm{T}}}
            \right)
            }{ \partial \mathbf{V}_{i,t} } \\
     &=
        \exp \left\{
            r_0
            - f \mathbf{V}_{i,t} \mathbf{C} \mathbf{V}_{i,t}^{\textrm{T}}
            - \alpha_0  \mathbf{\Omega}_{i,t} \,
                \textrm{e}^{- \mathbf{V}_{i,t} \mathbf{V}_{i,t}^{\textrm{T}}}
        \right\}
        \left[
            - 2 f \mathbf{V}_{i,t} \mathbf{C}
            - \alpha_0 \, \mathbf{\Omega}_{i,t} \,
                \textrm{e}^{- \mathbf{V}_{i,t} \mathbf{V}_{i,t}^{\textrm{T}}} \:
                \frac{\partial \! \left( - \mathbf{V}_{i,t} \mathbf{V}_{i,t}^{\textrm{T}} \right)
                    }{ \partial \mathbf{V}_{i,t} }
        \right] \\[2ex]
    \frac{ \partial F_{i,t} }{ \partial \mathbf{V}_{i,t} } &=
        \exp \left\{
            r_0
            - f \mathbf{V}_{i,t} \mathbf{C} \mathbf{V}_{i,t}^{\textrm{T}}
            - \alpha_0  \mathbf{\Omega}_{i,t} \,
                \textrm{e}^{- \mathbf{V}_{i,t} \mathbf{V}_{i,t}^{\textrm{T}}}
        \right\}
        \left[
            2 \alpha_0 \mathbf{\Omega}_{i,t} \,
                \textrm{e}^{- \mathbf{V}_{i,t} \mathbf{V}_{i,t}^{\textrm{T}}} \:
                \mathbf{V}_{i,t}
            - 2 f \mathbf{V}_{i,t} \mathbf{C}
        \right]
    \textrm{.}
\end{split}
\end{equation*}



Combining above with equation \ref{eq:trait-change}, we find that trait values at
time $t+1$ are

\begin{equation} \label{eq:trait-change-full}
    \mathbf{V}_{i,t+1} = \mathbf{V}_{i,t} + 2 \sigma_i^2
    \left(
        \alpha_0 \mathbf{\Omega}_{i,t} \,
            \textrm{e}^{- \mathbf{V}_{i,t} \mathbf{V}_{i,t}^{\textrm{T}}} \:
            \mathbf{V}_{i,t}
        - f \mathbf{V}_{i,t} \mathbf{C}
    \right)
    \textrm{.}
\end{equation}


\subsection*{Jacobian matrix}

The $nq \times nq$ Jacobian matrix consists of $n^2$ inner $q \times q$ blocks.
The on-diagonal blocks are the partial derivatives of species $i$ traits at time $t+1$ with respect
to species $i$ traits at time $t$:

\begin{equation*}
\begin{split}
    \frac{ \partial \; \mathbf{V}_{i,t+1} }{ \partial \; \mathbf{V}_{i,t} } &=
        \frac{ \partial \; \mathbf{V}_{i,t} }{ \partial \; \mathbf{V}_{i,t} } +
        2 \; \sigma_i^2
        \left(
            \frac{ \partial \;
                \alpha_0 \; \mathbf{\Omega}_{i,t} \;
                    \textrm{e}^{-\mathbf{V}_{i,t} \mathbf{V}_{i,t}^\textrm{T}} \,
                    \mathbf{V}_{i,t}}{\partial \; \mathbf{V}_{i,t} } -
            \frac{ \partial \; f \, \mathbf{V}_{i,t} \mathbf{C}}{\partial \; \mathbf{V}_{i,t} }
        \right) \\
    &=
        \mathbf{I} +
        2 \; \sigma_i^2
        \left[
            \alpha_0 \; \mathbf{\Omega}_{i,t} \,
            \left(
                \textrm{e}^{-\mathbf{V}_{i,t} \mathbf{V}_{i,t}^\textrm{T}} \: \mathbf{I} +
                \frac{ \partial \;
                        \textrm{e}^{-\mathbf{V}_{i,t} \mathbf{V}_{i,t}^\textrm{T}}
                        }{\partial \; \mathbf{V}_{i,t} } \, \mathbf{V}_{i,t}
            \right) -
            f \, \mathbf{C}
            \right] \\[2ex]
    \frac{ \partial \; \mathbf{V}_{i,t+1} }{ \partial \; \mathbf{V}_{i,t} } &= \mathbf{I} +
        2 \; \sigma_i^2 \;
        \left[
            \alpha_0 \; \mathbf{\Omega}_{i,t} \;
            \textrm{e}^{ - \mathbf{V}_{i,t} \mathbf{V}_{i,t}^{\textrm{T}} }
            \left(
                \mathbf{I} - 2 \; \mathbf{V}_{i,t}^{\textrm{T}} \mathbf{V}_{i,t}
            \right) -
            f \, \mathbf{C}
        \right]
    \textrm{,}
\end{split}
\end{equation*}

\noindent where $\mathbf{I}$ is a $q \times q$ identity matrix.


The off-diagonal blocks of the Jacobian are the partial derivatives of species $i$
traits at time $t+1$ with respect to species $k$ traits at time $t$, where $k \ne i$.
To calculate this, it's useful to rearrange equation \ref{eq:trait-change-full} and
extract the portion that includes $\mathbf{V}_{k,t}$:

\begin{equation*}
\begin{split}
    \mathbf{V}_{i,t+1} &= \mathbf{V}_{i,t} + 2 \; \sigma_i^2
    \left[
        \left(
            N_{k,t} \; \textrm{e}^{-\mathbf{V}_{k,t} \mathbf{D} \mathbf{V}_{k,t}^\textrm{T}} +
            \mathbf{\Phi}_{i,t}
        \right)
        \left(
            \alpha_0 \; \textrm{e}^{-\mathbf{V}_{i,t}
            \mathbf{V}_{i,t}^\textrm{T}} \; \mathbf{V}_{i,t}
        \right)
        - f \mathbf{V}_{i,t} \mathbf{C}
    \right] \\
    \mathbf{\Phi}_{i,t} &= N_{i,t} + \sum_{j \ne i, j \ne k}^{n}{
        N_{j,t} \; \textrm{e}^{- \mathbf{V}_{j,t} \mathbf{D}
        \mathbf{V}_{j,t}^{\textrm{T}}} }
    \textrm{.}
\end{split}
\end{equation*}

From this we calculated the partial derivative of $\mathbf{V}_{i,t+1}$ in relation to
$\mathbf{V}_{k,t}$


\begin{equation*}
\begin{split}
    \frac{ \partial \: \mathbf{V}_{i,t+1} }{ \partial \: \mathbf{V}_{k,t} } &=
        \frac{ \partial \: \mathbf{V}_{i,t} }{ \partial \: \mathbf{V}_{k,t} } +
        2 \; \sigma_i^2 \;
        \left[
            \frac{ \partial \:
                \left(
                    N_{k,t} \textrm{e}^{- \mathbf{V}_{k,t} \mathbf{D}
                    \mathbf{V}_{k,t}^{\textrm{T}}} + \mathbf{\Phi}_{i,t}
                \right)
                \left(
                    \alpha_0 \; \textrm{e}^{- \mathbf{V}_{i,t}
                    \mathbf{V}_{i,t}^{\textrm{T}}} \mathbf{V}_{i,t}
                \right)
            }{ \partial \:  \mathbf{V}_{k,t} } -
            \frac{ \partial \:  f \, \mathbf{V}_{i,t} \mathbf{C} }{
            \partial \: \mathbf{V}_{k,t} }
        \right] \\
    &= 2 \; \sigma_i^2 \; \alpha_0 \; N_{k,t} \;
        \frac{ \partial \:
                \textrm{e}^{
                    - \mathbf{V}_{k,t} \mathbf{D} \mathbf{V}_{k,t}^{\textrm{T}}
                    - \mathbf{V}_{i,t} \mathbf{V}_{i,t}^{\textrm{T}}
                } \; \mathbf{V}_{i,t}
            }{ \partial \:  \mathbf{V}_{k,t} } \\
    &= 2 \; \sigma_i^2 \; \alpha_0 \; N_{k,t} \;
        \frac{ \partial \:
                \left(
                    - \mathbf{V}_{k,t} \mathbf{D} \mathbf{V}_{k,t}^{\textrm{T}}
                    - \mathbf{V}_{i,t} \mathbf{V}_{i,t}^{\textrm{T}}
                \right)
            }{ \partial \:  \mathbf{V}_{k,t} } \;
        \textrm{e}^{
                    - \mathbf{V}_{k,t} \mathbf{D} \mathbf{V}_{k,t}^{\textrm{T}}
                    - \mathbf{V}_{i,t} \mathbf{V}_{i,t}^{\textrm{T}}
                } \; \mathbf{V}_{i,t} \\
    &= - 2 \; \sigma_i^2 \; \alpha_0 \; N_{k,t} \,
        \left( \mathbf{D} + \mathbf{D}^{\textrm{T}} \right) \,
        \mathbf{V}_{k,t}^{\textrm{T}} \;
        \textrm{e}^{
                    - \mathbf{V}_{k,t} \mathbf{D} \mathbf{V}_{k,t}^{\textrm{T}}
                    - \mathbf{V}_{i,t} \mathbf{V}_{i,t}^{\textrm{T}}
                } \; \mathbf{V}_{i,t} \\[2ex]
    \frac{ \partial \: \mathbf{V}_{i,t+1} }{ \partial \: \mathbf{V}_{k,t}} &=
        -4 \; \sigma_i^2 \; \alpha_0 \; N_{k,t} \;
        \mathbf{D} \; \mathbf{V}_{k,t}^{\textrm{T}} \;
        \textrm{e}^{
                - \mathbf{V}_{k,t} \mathbf{D} \mathbf{V}_{k,t}^{\textrm{T}}
                - \mathbf{V}_{i,t} \mathbf{V}_{i,t}^{\textrm{T}}
            } \;
            \mathbf{V}_{i,t}
    \textrm{.} \\
\end{split}
\end{equation*}



\subsection*{Keeping traits non-negative}

An obvious choice for keeping traits from being $<0$ is using absolute
values.
However, using absolute values causes fluctuations near zero
(because it "bounces off" the zero bound) that persist for a very long time.
This causes the simulations to take a prohibitively long time to reach
equilibrium.
The absolute value doesn't have a better-behaved derivative when traits are
zero than the method below, so that isn't an advantage.
Hence, we're not using absolute values.


The method we're using changes traits to zero when they would otherwise go
negative.
One term for this is a "Heaviside step function" ($H(x)$).
The derivative of this function is the Dirac delta function
($\delta(x)$).
They are defined as follows:

\begin{align}
    H(x) &= \begin{cases}
        0 & \text{if}\ x \le 0 \\
        1 & \text{if}\ x > 0
        \end{cases} \\
    \delta(x) &= \begin{cases}
        0 & \text{if}\ x \ne 0 \\
        \infty & \text{if}\ x = 0
        \end{cases} \\
\end{align}


Below, $\mathbf{\ddot{V}}(\mathbf{V}_t)$ can be
$\mathbf{V}_{i,t+1}(\mathbf{V}_{i,t})$ or
$\mathbf{V}_{i,t+1}(\mathbf{V}_{k,t})$ as shown above (i.e., without
any method to keep $\mathbf{V}_{i,t+1}$ from going negative).

\begin{align}
    \mathbf{\ddot{V}}(\mathbf{V}_t) &= \mathbf{V}_{t} + 2 \, \sigma^2
    \left(
        \alpha_0 \, \mathbf{\Omega}_{t} \,
            \textrm{e}^{- \mathbf{V}_{t} \mathbf{V}_{t}^{\textrm{T}}} \:
            \mathbf{V}_{t}
        - f \, \mathbf{V}_{t} \, \mathbf{C}
    \right) \\
    \frac{ \partial \; \mathbf{\ddot{V}}(\mathbf{V}_t) }{
        \partial \; \mathbf{V}_{t} } &=
        \mathbf{I} +
        2 \, \sigma^2
        \left[
            \alpha_0 \, \mathbf{\Omega}_{t} \,
            \textrm{e}^{ - \mathbf{V}_{t} \mathbf{V}_{t}^{\textrm{T}} }
            \left(
                \mathbf{I} - 2 \, \mathbf{V}_{t}^{\textrm{T}} \mathbf{V}_{t}
            \right) -
            f \, \mathbf{C}
        \right] \\
\end{align}



Here is how the main trait equation and its derivative work when including the
Heaviside step function:



\begin{align}
    \mathbf{V}_{t+1} &= H(\mathbf{\ddot{V}}(\mathbf{V}_t)) \;
        \mathbf{\ddot{V}}(\mathbf{V}_t) \\
    \frac{ \partial \, \mathbf{V}_{t+1}}{ \partial \, \mathbf{V}_{t} } &=
        \frac{ \partial \, H(\mathbf{\ddot{V}}(\mathbf{V}_t))}{
            \partial \, \mathbf{V}_{t} }
            \mathbf{\ddot{V}}(\mathbf{V}_t) +
        H(\mathbf{\ddot{V}}(\mathbf{V}_t))
        \frac{ \partial \, \mathbf{\ddot{V}}(\mathbf{V}_t) }{
            \partial \, \mathbf{V}_{t} } \\
    &= \delta(\mathbf{\ddot{V}}(\mathbf{V}_t))
        \frac{ \partial \, \mathbf{\ddot{V}}(\mathbf{V}_t) }{
            \partial \, \mathbf{V}_{t} }
        \mathbf{\ddot{V}}(\mathbf{V}_t) +
        H(\mathbf{\ddot{V}}(\mathbf{V}_t))
        \frac{ \partial \, \mathbf{\ddot{V}}(\mathbf{V}_t) }{
            \partial \, \mathbf{V}_{t} } \\
    &= \frac{ \partial \, \mathbf{\ddot{V}}(\mathbf{V}_t) }{
            \partial \, \mathbf{V}_{t} }
        \left[
            \delta(\mathbf{\ddot{V}}(\mathbf{V}_t))
            \mathbf{\ddot{V}}(\mathbf{V}_t) +
            H(\mathbf{\ddot{V}}(\mathbf{V}_t))
        \right]
\end{align}



So by including the Heaviside step function, we can multiply the "normal"
derivative (the ones above) by the term
$\delta(\mathbf{\ddot{V}}(\mathbf{V}_t)) \mathbf{\ddot{V}}(\mathbf{V}_t) +
H(\mathbf{\ddot{V}}(\mathbf{V}_t))$ to get the updated derivative.
Based on the definitions of $H(x)$ and $\delta(x)$...

\begin{itemize}
    \item If $\mathbf{\ddot{V}}(\mathbf{V}_t)$ is 0 at equilibrium, then
        $\partial \mathbf{V}_{t+1} / \partial \mathbf{V}_{t}$
        is undefined (because we're multiplying $\infty$ by zero).
    \item If $\mathbf{\ddot{V}}(\mathbf{V}_t)$ is $> 0$ at equilibrium, then
        $\partial \mathbf{V}_{t+1} / \partial \mathbf{V}_{t}$
        simplifies to $\mathbf{\ddot{V}}(\mathbf{V}_t)$.
    \item If $\mathbf{\ddot{V}}(\mathbf{V}_t)$ is $< 0$ at equilibrium, then
        $\partial \mathbf{V}_{t+1} / \partial \mathbf{V}_{t}$
        simplifies to zero.
\end{itemize}


